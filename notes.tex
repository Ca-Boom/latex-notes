\documentclass[a4paper,12pt]{article}

\usepackage[margin=2.5cm]{geometry}
\usepackage{amsmath}
\usepackage{graphicx}
\usepackage{xcolor}
\usepackage{soul}
\usepackage{lipsum}
\usepackage[utf8]{inputenc}  % Allow UTF-8 input
\usepackage{tcolorbox}  % For making a highlighted block of texts

% This sets space between paragraphs to one line width
\setlength{\parskip}{\baselineskip}

% This sets the paragraph indentation to zero (no indents)
\setlength\parindent{0pt}

\begin{document}

\begin{titlepage}
    \centering
    \vspace*{3cm}  % Need * to skip at the top of the page
    {\scshape \Huge Notes on \LaTeX \par}
    \vspace{6cm}
    {\LARGE \itshape By Ca'Boom \par}
    \vspace{6cm}
    {\Large Last Updated: \today \par}
    \vfill

\end{titlepage}

\section{Introduction}

This document contain various tips and tricks on writing in \LaTeX.

To contribute, simply add stuff :). Use \verb|\verb| to print in-line \LaTeX commands as normal texts. For section heading, use a combination of \verb|\textt| and \verb|\textbackslash| instead as \verb|\verb| will not work.

If you have a write permission to this document on \texttt{Overleaf}, you can sync your updates to the \texttt{GitHub} repository directly from the \texttt{Overleaf} menu. Simply select \emph{GitHub} under \emph{Sync} to sync.

Sectional organisational of this document is TBD and will change as the document grow.

\section{Outlining}

\subsection{\texttt{\textbackslash paragraph} command}
\LaTeX provide seven layers of sectioning commands.

\begin{enumerate}
    \item \verb|\part|
    \item \verb|\chapter|
    \item \verb|\section|
    \item \verb|\subsection|
    \item \verb|\subsubsection|
    \item \verb|\paragraph|
    \item \verb|\subparagraph|
\end{enumerate}

\verb|\paragraph| provides headings for paragraphs, which are not usually needed in normal writing. However, the paragraph headings are handy for outlining your document as in an example below.

\begin{tcolorbox}
    \paragraph{heading} \lipsum[6] 
\end{tcolorbox}



\end{document}